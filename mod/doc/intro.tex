% -*- mode: LaTeX; coding: utf-8; -*-

\section*{Введение}
\addcontentsline{toc}{section}{Введение}
\label{sec:intro}

Одной из основных и самых больших частей компиляторов и
интерпретаторов современных языков программирования является
синтаксический анализ исходных файлов. Анализатором производится
разбор исходного кода программы сначала на лексемы, а затем на слова и
словосочетания. Из полученных слов формируется дерево разбора. На этом
работа анализатора заканчивается и в дело вступают другие модули
компилятора.

Порой бывает сложно представить, что анализаторы поддерживают весь
невообразимо огромный синтаксис современных языков программирования,
но это действительно так.  Синтаксический анализ может применяться не
только, как программы разбора исходного кода. С помощью него можно
анализировать любые текстовые данные, использующие предварительно
описанную нотацию. Это могут быть как различные общепринятые нотации
(например, используемые в математике), так и конфигурационные файлы
приложений (fetchmail).

%%% Local Variables: 
%%% mode: latex
%%% TeX-master: "main"
%%% End: 
