% -*- mode: LaTeX; coding: utf-8; -*-

\section*{Заключение}
\addcontentsline{toc}{section}{Заключение}
\label{sec:conclusion}

В процессе работы над курсовым проектом, была разработана программа
разбора математических выражений. Возможно использование следующих
математических выражений: +, -, *, /, sin, cos, tan, asin, acos,
atan. Также допустимо объединять выражения в различные математически
правильные синтаксические конструкции с использованием скобочек '(',
')' для задания очередности выполнения вычислений.

При генерации парсера Bison предупреждает о двадцати конфликтах
сдвига/вывода.  Т.е. он предупреждает о возможных разногласиях между
введенными пользователем командами и произведенными парсером
действиями.  Но на практике это не достовляетя проблем, так как
подобные действия перехватываются лексическим анализатором перед
парсером и выводятся ошибки в консоль пользователя.

Еще одна особенность разработанной программы~--- возможность
применения в многозадачном приложении и возможность работы с потоками
информации.


%%% Local Variables: 
%%% mode: latex
%%% TeX-master: "main"
%%% End: 
