% -*- mode: LaTeX; coding: utf-8; -*-

\section{Аналитическая часть}
\label{sec:theory}

\subsection{Лексический анализатор}
\label{sec:lexic}

Лексический анализ~--— процесс аналитического разбора входной
последовательности символов~(например, такой как исходный код на одном
из языков программирования) с целью получения на выходе
последовательности символов, называемых «токенами»~(подобно
группировке букв в словах).  При этом, группа символов входной
последовательности, идентифицируемая на выходе процесса как токен,
называется лексема, то есть в процессе лексического анализа
производится распознавание и выделение лексем из входной
последовательности символов~\cite{lex_wiki}.

Как правило, лексический анализ производится с точки зрения
определенного языка или набора языков.  Язык, а точнее его грамматика,
задает определенный набор лексем, которые могут встретиться на входе
процесса.

Традиционно принято организовывать процесс лексического анализа,
рассматривая входную последовательность символов, как поток символов.
При такой организации, процесс самостоятельно управляет выборкой
отдельных символов из входного потока.

Основная задача лексического анализа~--- разбить входной текст,
состоящий из последовательности одиночных символов, на
последовательность слов, или лексем, т.е. выделить эти слова из
непрерывной последовательности символов.  Все символы входной
последовательности с этой точки зрения разделяются на символы,
принадлежащие каким-либо лексемам, и символы, разделяющие
лексемы~(разделители).  В некоторых случаях между лексемами может и не
быть разделителей.  С другой стороны, в некоторых языках лексемы могут
содержать незначащие символы~(пробел в Фортране).  В С разделительное
значение символов-разделителей может блокироваться~('' в конце строки
внутри "...")~\cite{lex}.

Обычно все лексемы делятся на классы.  Примерами таких классов
являются числа~(целые, восьмеричные, шестнадцатеричные, действительные
и т.д.), идентификаторы, строки.  Отдельно выделяются ключевые слова и
символы пунктуации~(иногда их называют символы-ограничители).  Как
правило, ключевые слова~--- это некоторое конечное подмножество
идентификаторов.  В некоторых языках~(например, ПЛ/1) смысл лексемы
может зависеть от ее контекста и невозможно провести лексический
анализ в отрыве от синтаксического.

С точки зрения дальнейших фаз анализа лексический анализатор выдает
информацию двух сортов: для синтаксического анализатора, работающего
вслед за лексическим, существенна информация о последовательности
классов лексем, ограничителей и ключевых слов, а для контекстного
анализа, работающего вслед за синтаксическим, важна информация о
конкретных значениях отдельных лексем~(идентификаторов, чисел и т.д.).
Поэтому общая схема работы лексического анализатора такова.  Сначала
выделяем отдельную лексему~(возможно, используя символы-разделители).
Если выделенная лексема-ограничитель, то он (точнее, некоторый его
признак) выдается как результат лексического анализа.  Ключевые слова
распознаются либо явным выделением непосредственно из текста, либо
сначала выделяется идентификатор, а затем делается проверка на
принадлежность его множеству ключевых слов.  Если да, то выдается
признак соответствующего ключевого слова, если нет~--- выдается признак
идентификатора, а сам идентификатор сохраняется отдельно.  Если
выделенная лексема принадлежит какому-либо из других классов
лексем~(число, строка и т.д.), то выдается признак класса лексемы, а
значение лексемы сохраняется.

Лексический анализатор может работать или как самостоятельная фаза
трансляции, или как подпрограмма, работающая по принципу "дай
лексему".  В первом случае выходом лексического анализатора является
файл лексем, во втором лексема выдается при каждом обращении к
лексическому анализатору~(при этом, как правило, тип лексемы
возвращается как значение функции "лексический анализатор", а значение
передается через глобальную переменную).  С точки зрения формирования
значений лексем, принадлежащих классам лексем, лексический анализатор
может либо просто выдавать значение каждой лексемы и в этом случае
построение таблиц переносится на более поздние фазы, либо он может
самостоятельно строить таблицы объектов~(идентификаторов, строк, чисел
и т.д.).  Тогда в качестве значения лексемы выдается указатель
на вход в соответствующую таблицу.

\subsection{Синтаксический анализатор}
\label{sec:syntax}

Синтаксический анализ~(парсинг)~--— это процесс сопоставления линейной
последовательности лексем~(слов, токенов) языка с его формальной
грамматикой.  Результатом обычно является дерево разбора
(синтаксическое дерево).  Обычно применяется совместно с лексическим
анализом.  Синтаксический анализатор~(парсер)~--— это программа или
часть программы, выполняющая синтаксический анализ~\cite{syntax_wiki}.

При парсинге исходный текст преобразуется в структуру данных,
обычно~--— в дерево, которое отражает синтаксическую структуру входной
последовательности и хорошо подходит для дальнейшей обработки.

Как правило, результатом синтаксического анализа является
синтаксическая структура предложения, представленная либо в виде
дерева зависимостей, либо в виде дерева составляющих, либо в виде
некоторой комбинации первого и второго способов представления.

GNU Bison~--— программа, предназначенная для автоматического создания
синтаксических анализаторов по данному описанию
грамматики~\cite{bison}.

Для того, чтобы Bison мог разобрать программу на каком-то языке, этот
язык должен быть описан контекстно-свободной грамматикой.  Это
означает, что вы определяете одну или более синтаксических групп и
задаете правила их сборки из составных частей.  Например, в языке C
одна из групп называется `выражение'.  Правило для составления
выражения может выглядеть так: "Выражение может состоять из знака
`минус' и другого выражения".  Другое правило: "Выражением может быть
целое число".  Как вы может видеть, правила часто бывают рекурсивными,
но должно быть по крайней мере одно правило, выводящее из рекурсии.

Самой распространенной формальной системой является форма
Бэкуса-Наура~(БНФ, Backus-Naur Form, BNF), которая была разработана
для представления таких правил в удобном для человека виде и для
описания языка Algol~60.  Любая грамматика, выраженная в форме
Бэкуса-Наура является контекстно-свободной грамматикой.  Bison
принимает на вход, в сущности, особый вид БНФ, адаптированный для
машинной обработки.

Bison может работать не со всеми контекстно-свободными грамматиками, а
только с грамматиками класса LALR.  Коротко, это означает, что должно
быть возможно определить, как разобрать любую часть входа, заглядывая
вперед не более, чем на одну лексему.  Строго говоря, это описание
LR-грамматики, класс LALR имеет дополнительные ограничения, которые не
так просто объяснить.  Но в обычной практике редко встречаются
LR-грамматики, которые не являются LALR.

%%% Local Variables: 
%%% mode: latex
%%% TeX-master: "main"
%%% End: 



