% -*- mode: LaTeX; coding: utf-8; -*-

\section{Проектная часть}
\label{sec:project}

\subsection{Выбор инструмента}
\label{sec:utils}

Для реализации поставленной задачи были выбраны следующии инструменты:
\begin{enumerate}
\item Ubuntu~9.04~\cite{ubuntu_home}~--- операционная система;
\item GNU Emacs~\cite{emacs_home}~--- редактор;
\item GNU gcc~\cite{gcc,gcc_home}~--- компилятор языка С~\cite{k_r_c};
\item GNU make~\cite{make,make_home}~--- программа управления
  компиляцией;
\item Bison~\cite{bison,bison_home}~--- генератор синтаксических
  анализаторов;
\item Flex~\cite{lex,flex_home}~--- генератор лексических
  анализаторов;
\item MinGW~\cite{mingw}~--- портированная версия компилятора gcс,
  заголовочных файлов и набора библиотек для операционных систем
  семейства Microsoft Windows.
\end{enumerate}
Были выбраны именно эти иснтрументы, т.к. часть из них достаточно
давно и успешно применяются для решения задач синтаксического
разбора~(Flex и Bison) и все они распространяются на условиях
свободной лицензии.  Для семейства операционных систем, к которому
принадлежит выбранная, очень просто установить все необходимые
программы с помощью менеджера пакетов.

\subsection{Выполнение проекта}
\label{sec:execute}

Для начала был написан простейший главный файл для всего
приложения~--- main.c~(листинг~\ref{list:main_c}) и заголовочный файл
к нему~--- main.h~(листинг~\ref{list:main_h}).  Их основная цель~---
запуск синтаксического анализатора.

Далее был составлен Makefile~(листинг~\ref{list:makefile}).  В нем
были отражены основные условия компиляции будущего приложения.  Для
обеспечения большой переносимости как между отдельными версиями
программы GNU make, так и между различными операционными системами
было решено использовать минимальное количество явных правил,
реализацию управления компиляцией через дополнительные переменные,
условные цепочки правил для генерации и компиляции лексического и
синтаксического генераторов.

В качестве синтаксиса приложения был использован математически
правильный язык, содержащий основные арифметические и
тригонометрические действия.  Такие как, сложение, вычитание,
умножени, деление, синус, косинус, тангенс, арксинус, арккосинус,
арктангенс, а также определины лексемы для открывающей и закрывающей
скобок.  Разработанная нотация лексем была записана в файл
source.l~(листинг~\ref{list:source}) для последующей генерации
лексического анализатора с помощью программы Flex.  При компиляции
сгенерированного исходного кода можно было получить лексический
анализатор, способный распознавать заложенные в нотации лексемы~---
алгебраические знаки, тригонометрические функции и скобки.

Следующим на очереди был~--- y.y~(листинг~\ref{list:y}).  В нем
описана грамматика для синтаксического анализатора.  Были
сформулированы действия, которые необходимо выполнить при поступлении
определенных лексем от лексического анализатора.  На этом этапе в
source.l были добавлены несколько служебных лексем для предупреждения
о не распознанных лексемах, иначе он просто офильтрововал не
распознанные лексемы, никак об этом не сигнализируя.  Обработка
служебных лексем выполняется синтаксическим анализатором.  Полученный
файл передается программе Bison для генерации исходного кода парсера.

%%% Local Variables: 
%%% mode: latex
%%% TeX-master: "main"
%%% End: 
